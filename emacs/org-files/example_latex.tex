% Created 2022-07-23 Sat 16:07
% Intended LaTeX compiler: pdflatex
\documentclass[a4]{article}
\usepackage[utf8]{inputenc}
\usepackage[T1]{fontenc}
\usepackage{graphicx}
\usepackage{longtable}
\usepackage{wrapfig}
\usepackage{rotating}
\usepackage[normalem]{ulem}
\usepackage{amsmath}
\usepackage{amssymb}
\usepackage{capt-of}
\usepackage{hyperref}
\author{Reikimann}
\date{\today}
\title{TuxDocs}
\hypersetup{
 pdfauthor={Reikimann},
 pdftitle={TuxDocs},
 pdfkeywords={},
 pdfsubject={},
 pdfcreator={Emacs 28.1 (Org mode 9.5.2)}, 
 pdflang={English}}
\begin{document}

\maketitle
\tableofcontents

\begin{verbatim}

<h2 align="center">TuxDocs</h2>

  <p align="center">
    A quickstart guide for linux that also goes in-depth for advanced users  
    <br />
    <br />
    <a href="https://github.com/Reikimann/TuxDocs/issues">Report Bug</a>
    ·
    <a href="https://github.com/Reikimann/TuxDocs/issues">Request Feature</a>
    ·
    <a href="https://github.com/Reikimann/TuxDocs/pulls">Add to the project</a>
  </p>
</p>

<p align="center">
<img src=https://img.shields.io/github/stars/Reikimann/TuxDocs?style=for-the-badge&logo=appveyor&color=blue/>
<img src=https://img.shields.io/github/forks/Reikimann/TuxDocs?style=for-the-badge&logo=appveyor&color=blue/>
<img src=https://img.shields.io/github/issues/Reikimann/TuxDocs?style=for-the-badge&logo=appveyor&color=informational/>
<img src=https://img.shields.io/github/issues-pr/Reikimann/TuxDocs?style=for-the-badge&logo=appveyor&color=informational/>
</p>
<br />

\end{verbatim}

Well hello there. This project was started as a discord-server for linux users (mostly at our school), because we (Logicguy1 \& Reikimann) wanted to get people to transfer to linux. We realized that people don't want to do things that seems complicated, so that's where this project came into play.

We needed a way to introduce people to linux, that wouldn't want to watch hours of linux youtube content. So I, Reikimann, came up with the idea to write this massive, chapter based leveling documentation, to introduce people to linux, some cool stuff and other ideas.

For now, the release schedule of this project can't be determined, because of our school. We intent to release one chapter/level at a time to the main branch, but you are welcome to watch and contribute to the project through the Readme-edits branch.

\href{https://github.com/Reikimann/TuxDocs}{Github Test-link}

\section*{Chapter 1 - Getting started}
\label{sec:org477810f}

During this chapter we will go over some of the following items:

\begin{itemize}
\item \href{./Chapter\_1/distros.md}{Distros and flavors, choosing your first distro}
\item \href{./Chapter\_1/technical\_terminologies.md}{Technical terminologies}
\item \href{./Chapter\_1/getting\_comfortable.md}{Getting comfortable with the commandline}
\begin{itemize}
\item Managing package managers
\end{itemize}
\item \href{./Chapter\_1/setting\_up.md}{Setting up your system}
\item \href{./Chapter\_1/system\_configs.md}{System configs}
\item \href{./Chapter\_1/GUI\_programs.md}{Useful programs - GUI}
\item \href{./Chapter\_1/CLI\_programs.md}{CLI Utils (command line interface)}
\item \href{./Chapter\_1/KDE\_connect.md}{KDE Connect}
\item \href{./Chapter\_1/browsers.md}{Browsers, search engines \& plugins}
\end{itemize}

To read chapter 1 \href{./Chapter\_1/README.md}{here}.

\section*{Chapter 2}
\label{sec:org61b8627}

\ldots{}

\section*{Contributing}
\label{sec:orgad88ff0}

This is a new project and we do not claim to be experts on this topic. We would simply love more people joining the Linux community and we believe developing this resource could help newcomers and long time users alike. So if you would like to help us extending this resource, wether it being correcting spelling mistakes or writing new articles, you are more than welcome. Please see the \href{./CONTRIBUTING.org}{contributing} file for more information.
\end{document}